\subsection{The missing OS Abstraction: sel4}
In the paper Time Protection: The missing OS Abstraction
\cite{MissingOSAbstraction} a customized version of the sel4 microkernel is
presented, which aims to protect against all possible timing channels via. the
OS.

\subsubsection{Threat model}
In the paper they present the threat model as being a victim program, where a
Trojan within the program uses any means necessary to attempt to leak
information through timing channels. There are no restrictions on the Trojan
outside it simply running in a separate domain than that of the receiver/spy
program.

Specifically they differentiate between two types of channels, namely the
\textbf{covert} and \textbf{side} channel. The covert channel requires the
collusion between the sender and the receiver and represents the worst-case
scenario, where the sender intentionally attempts to transmit information. The
side channel is instead when the sender leaks information to the receiver
unwittingly. That is, the sender leaks information via normal program
execution. By protecting against the covert channel it immediately follows,
that the side channel is protected.


\subsubsection{Channels}
Two main types of channels are referenced throughout the paper. The on-core
state, including the L1, L2 caches, Translation Lookaside Buffers and more,
which are unique to each core on the CPU and the shared state which includes
higher level caches that are shared between different cores. To protect against
all, they provide the following five core requirements:

\begin{itemize}
  \item \emph{Flush on-core state:} When time-sharing a core, all on-core
    microarchitectural state must be flushed on domain switch, unless the
    hardware supports specific partitioning of this state.
  \item \emph{Partition the OS:} Each security domain must have a private copy
    of OS text, stack and global data. That is, all the dynamic OS kernel
    memory must have the same separation as userland execution.
  \item \emph{Deterministic data sharing:} Access to any remaining shared OS
    data must be sufficiently deterministic to avoid information leakage.
  \item \emph{Flush deterministically:} State flushing must be padded to its worst case latency.
  \item \emph{Partition interrupts:} When sharing a core, the OS must disable
    or partition any interrupts other than the preemption timer. (This is only a concern intra-core).
\end{itemize}


\subsubsection{Cloning the OS}
In the paper they introduce a method for cloning the OS. A userland program,
which has the capability of cloning gives up a subset of it's own allocated
memory for the clone call, and the kernel then creates a kernel image copy in
the user-supplied memory. This memory now gets marked as kernel data, meaning
that the userland can not change it, but that any syscalls will run through
this kernel copy.

This works via the security domains of sel4. The initial kernel creates a
master Kernel\_Image which represents the present kernel and includes the clone
right. It hands this capability along with the size of the image to the initial
user process, which then partitions the memory into security domains, where it
clones the OS into each of these new security domains. 

This cloning consists of three steps:
\begin{enumerate}
  \item The user process retypes some Untyped memory into an uninitialized Kernel\_Image and Kernel\_Memory of sufficient size.
  \item it allocates address space identifier to the uninitialised Kernel\_Image.
  \item it invokes the Kernel\_Clone on the new Kernel\_Image.
\end{enumerate}

\subsubsection{Cache coloring}
To protect the caches that are too large to be flushed, cache coloring is
implemented. This means, that separate security domains are unable to access
the same parts of caches, such that they have no influence on each other. This
reduces the total amount of available cache space for each domain, and it means
that the process which creates each security domain has to take this factor
into account, such that the distribution of the cache space is sufficient for
all domains.
